To unære prædikater $S$ og $Æ$ er givet, samt fortolkningen $\mathcal{F}$. \\

dom($\mathcal{F}$) $=$ alle mennesker.\\
$S^{\mathcal{F}}=$ \_ er studerende. \\
$Æ^{\mathcal{F}}=$ \_ er ærlig. \\
\\
Følgende sætninger er oversat til prædikatlogik under fortolkningen $\mathcal{F}$:\\
\\
a) Alle studerende er ærlige:\\
    $$\forall x(S(x) \rightarrow Æ(x))$$

b) Alle studerende er uærlige:\\
    $$\forall x(S(x) \rightarrow \neg Æ(x))$$
    
c) Ikke alle studerende er ærlige:\\
    $$\neg\forall x(S(x) \rightarrow Æ(x))$$
    
d) Ikke alle studerende er uærlige:\\
    $$\neg\forall x(S(x) \rightarrow \neg Æ(x))$$
    
d) Nogen studerende er ærlige:\\
    $$ \exists x(S(x) \wedge Æ(x))$$

f) Nogen studerende er uærlige:\\
    $$ \exists x(S(x) \wedge \negÆ(x))$$
    
g) Ingen studerende er ærlige:\\
    $$ \neg\exists x(S(x) \wedge Æ(x))$$
    
h) Ingen studerende er uærlige:\\
    $$ \neg\exists x(S(x) \wedge \negÆ(x))$$
    
Vi kan se, at fx. udtrykket i g) og i b) indeholder samme information, hvilket antyder at man kan gå fra en alkvantor til en eksistenskvantor ved at negere udtrykket (samt forholde sig til $\rightarrow$ og $\wedge$).